% Created 2024-09-23 Mon 13:35
% Intended LaTeX compiler: pdflatex
%%% TeX-command-extra-options: "-shell-escape"

\documentclass{iacrtrans}
\usepackage[utf8]{inputenc}
\usepackage[T1]{fontenc}

% -- Default Packages --
\usepackage{graphicx}
\usepackage{longtable}
\usepackage{wrapfig}
\usepackage{rotating}
\tcbuselibrary{skins}
\usepackage{booktabs}
% \usepackage{multirow}
\usepackage[normalem]{ulem}
\usepackage{amsmath}
\usepackage{amssymb}
\usepackage{capt-of}
\usepackage{hyperref}
\usepackage[nameinlink]{cleveref}
\crefname{algocf}{alg.}{algs.}
\Crefname{algocf}{Algorithm}{Algorithms}
\usepackage{xstring}
\usepackage{tikz}
\usetikzlibrary{shapes.geometric, arrows.meta, positioning, calc}

\newtcolorbox{empheqboxed}{
  enhanced,
  boxsep=1pt,
  arc=0.75ex,
  colback=gray!10,
  colframe=gray!40,
  boxrule=1pt,
  leftrule=40pt,
  top=-3.5mm,
  overlay unbroken and first ={%
    \node[minimum width=1cm,
      anchor=south,
      font=\sffamily\bfseries,
      xshift=20pt,
      yshift=-6.5pt,
    black]
    at (frame.west) {Script:};
  }
}

\newcommand{\mycomment}[1]{}
\newcommand{\elem}[1]{\, \langle #1 \rangle \,}
\newcommand{\opcode}[1]{\, \texttt{#1} \,}
\newcommand{\script}[1]{ $\big\{ #1 \big\}$ }
\newcommand{\nero}{$\mathcal{N}\mathfrak{e}\mathcal{R}O$}

% -- Algorithms --
\usepackage[
  titlenumbered,
  linesnumbered,
  ruled
]{algorithm2e}
\SetKwInOut{Input}{Input}
\SetKwInOut{Output}{Output}
\SetKwInOut{Return}{Return}

\SetKwComment{Comment}{/* }{ */}

\DeclareMathOperator*{\argmax}{arg\,max}
\DeclareMathOperator*{\argmin}{arg\,min}

\author{Oleksandr Kurbatov\inst{1} \and Dmytro Zakharov\inst{1}
\and Kyrylo Baibula \inst{1}}
\institute{Distributed Lab
  \email{ok@distributedlab.com},
  \email{dmytro.zakharov@distributedlab.com},
\email{kyrylo.baybula@distributedlab.com}}
\title[Verifiable Computation on
Bitcoin]{\nero: BitVM2-Based Generic
Optimistic Verifiable Computation on Bitcoin}

\hypersetup{
  pdfauthor={Distributed Lab},
  pdftitle={BitVM2 Made Practical},
  pdfkeywords={},
  pdfsubject={},
  pdfcreator={Emacs 29.4 (Org mode 9.7.11)},
  pdflang={English}
}

\def\bitcoin{%
  \leavevmode
  \vtop{\offinterlineskip
    \setbox0=\hbox{B}%
    \setbox2=\hbox to\wd0{\hfil\hskip-.03em
    \vrule height .3ex width .15ex\hskip .08em
    \vrule height .3ex width .15ex\hfil}
    \vbox{\copy2\box0}\box2}}


\usepackage{biblatex}
\addbibresource{refs.bib}

\begin{document}

\maketitle

\keywords[]{Bitcoin, Bitcoin Script, BitVM2, Verifiable Computation,
Optimistic Verification, L2 Layer, Zero-Knowledge Proofs}

\begin{abstract}
  One of Bitcoin's biggest unresolved challenges is the ability to execute a
  large arbitrary program on-chain. Namely, publishing a program written in
  Bitcoin Script that exceeds 4 MB is practically impossible. This is a strict
  restriction as, for instance, it is impossible to multiply two large integers,
  not even mentioning a zero-knowledge proof verifier. To address this issue, we
  narrow down the problem to the verifiable computation which is more feasible
  given the current state of Bitcoin.

  One of the ways to do it is the \textit{BitVM2} protocol. Based on
  it, we are aiming to
  create a generic library for the on-chain verifiable computations. This
  document is designated to state our progress, pitfalls, and challenges
  encountered during the development. While most of the current
  efforts are put into
  transferring the \textit{Groth16} verifier on-chain with the
  main focus on implementing bridge, we try to solve
  a broader problem, enabling a more significant number of potential use cases
  (including zero-knowledge proofs verification).
\end{abstract}

\setcounter{tocdepth}{2}
\tableofcontents

\begin{tcolorbox}[colback=green!15!white, halign title=flush center,
  colframe=green!70!black, fonttitle=\bfseries\large, title=Note, sharp corners]
  \centering This is a very early version of the paper, development is still in
  active progress!
\end{tcolorbox}

\section{Introduction}\label{sec:intro}

The Bitcoin Network \autocite{bitcoin_paper} is rapidly growing. However, the
Bitcoin Script, the native programming language of Bitcoin, imposes strict size
limits on transactions --- only 4 MB are allowed, making it challenging to
implement any advanced cryptographic (and not only) primitives, among which
highly desirable zero-knowledge proofs verification on-chain. To address this
limitation, the \textit{BitVM2} \autocite{bitvm2} proposal introduces
an innovative
approach that enables the optimistic execution of large programs on the Bitcoin
chain.

The proposed method suggests that the executor (which is called an
\textbf{operator}) splits the large program into smaller chunks (which we
further refer to as \textbf{shards}) and commits to the intermediate values.
This way, if the computation is incorrect, it must be incorrect in some shard,
and it can be proven \textit{concisely} due to the splitting mechanism.

This document provides a concise overview of our progress in implementing the
library for generic, optimistic, verifiable computation on Bitcoin. Currently,
we are focusing on reproducing the \textit{BitVM2} paper approach
while not limiting the
function and input/output format as much as possible.

\section{Program Split}\label{sec:program-splitting}

\subsection{Public Verifiable Computation}

Since the main goal of our research is to build the \textit{public verifiable
computation}, it is reasonable to start with a brief overview of this concept. A
\textit{public verifiable computation scheme} allows the (potentially)
computationally limited verifier $\mathcal{V}$ outsource the evaluation of some
function $f$ on input $x$ to the prover (worker) $\mathcal{P}$. Then,
$\mathcal{V}$ can verify the correctness of the provided output $y$ by
performing significantly less work than $f$ requires.

In the context of Bitcoin on-chain verification, $\mathcal{V}$ can be viewed as
the Bitcoin smart contract which is heavily limited in computational resources
(due to the inherit Bitcoin Script inexpressiveness). The prover $\mathcal{P}$
is the operator who executes the program on-chain. The program $f$ is the
Bitcoin Script, and the input $x$ is the data provided by the operator.

Now, we define the \textit{public verifiable computation scheme} as follows:
\begin{definition}
  A public verifiable computation (VC) scheme $\Pi_{\text{VC}}$
  consists of three probabilistic polynomial-time algorithms:
  \begin{itemize}
    \item $\textsc{Gen}(f,1^{\lambda})$: randomized algorithm, taking the
      security parameter $\lambda \in \mathbb{N}$ and the function $f$ as input,
      and outputting the prover and verifier parameters $\mathsf{pp}$ and
      $\mathsf{vp}$.
    \item $\textsc{Compute}(\mathsf{pp}, x)$: deterministic algorithm, taking
      the prover parameters $\mathsf{pp}$ and the input $x$, and
      outputting $y=f(x)$
      together with a ``proof of computation'' $\pi$.
    \item $\textsc{Verify}(\mathsf{vp}, x, y, \pi)$: given the verifier
      parameters $\mathsf{vp}$, the input $x$, the output $y$, and the proof
      $\pi$, the algorithm outputs $\mathsf{accept}$ or
      $\mathsf{reject}$ based on
      the correctness of the computation.
  \end{itemize}

  Such scheme should satisfy the following properties (informally):
  \begin{itemize}
    \item \textbf{Correctness}. Given any function $f$ and input $x$,
      \begin{equation*}
        \text{Pr}\left[\mathsf{Verify}(\mathsf{vp}, x, y, \pi) =
          \mathsf{accept}\; \Big| \;
          \begin{matrix}
            (\mathsf{pp},\mathsf{vp}) \gets \mathsf{Gen}(f,1^{\lambda}) \\
            (y,\pi) \gets \mathsf{Compute}(\mathsf{pp},x)
        \end{matrix}\right] = 1
      \end{equation*}
    \item \textbf{Security}. For any $f$ and any probabilistic
      polynomial-time adversary $\mathcal{A}$,
      \begin{equation*}
        \text{Pr}\left[\mathsf{Verify}(\mathsf{vp}, \widetilde{x},
          \widetilde{y}, \widetilde{\pi}) = \mathsf{accept}\; \Big| \;
          \begin{matrix}
            (\mathsf{pp},\mathsf{vp}) \gets \mathsf{Gen}(f,1^{\lambda}) \\
            (\widetilde{x}, \widetilde{y}, \widetilde{\pi}) \gets
            \mathcal{A}(\mathsf{pp}, \mathsf{vp}), \;
            f(\widetilde{x}) \neq \widetilde{y}
        \end{matrix}\right] \leq \mathsf{negl}(\lambda)
      \end{equation*}
    \item \textbf{Efficiency}. $\mathsf{Verify}$ should be much cheaper than the
      evaluation of $f$. For example, if the evaluation of $f$ takes $T_f$,
      then the verification could take $T_{\mathcal{V}} =
      \mathcal{O}(\log T_f)$.
  \end{itemize}
\end{definition}

\subsection{Motivation for Verifiable Computation on Bitcoin}
Suppose we have a large program $f$ implemented inside the Bitcoin Script and
want to verify its execution on-chain. Suppose the prover $\mathcal{P}$ claims
that ${y} = f({x})$ for published ${x}$ and ${y}$. Some of the examples include:
\begin{itemize}
  \item \textbf{Field Multiplication}: $f(a,b) = a \times b$ for $a,b \in
    \mathbb{F}_p$. Here, the input ${x}=(a,b) \in \mathbb{F}_p^2$ is a tuple of
    two field elements, while the output $y \in \mathbb{F}_p$ is a single field
    element.
  \item \textbf{EC Points Addition}: $f(x_1,y_1,x_2,y_2) = (x_1,y_1) \oplus
    (x_2,y_2) = (x_3,y_3)$. Input is a tuple $(x_1,y_1,x_2,y_2)$ of four field
    elements, representing the coordinates of two elliptic curve points. The
    output is a point $(x_3,y_3)$, represented by two field elements
    $\mathbb{F}_p$.
  \item \textbf{Groth16 Verifier}: $f(\pi_1,\pi_2,\pi_3) = b$ for $b \in
    \{\mathsf{accept}, \mathsf{reject}\}$. Based on three provided points
    $\pi_1$,$\pi_2$,$\pi_3$, representing the proof, decide whether the proof is
    valid.
\end{itemize}

As mentioned before, publishing $f$ entirely on-chain is not an option. Instead,
the \textit{BitVM2} paper suggests splitting the program into shards
(subprograms) $f_1,\dots,f_n$ such
that $f=f_n \circ f_{n-1} \circ \dots \circ f_1$, where $\circ$ denotes the
function composition. This way, both the prover $\mathcal{P}$ and verifier
$\mathcal{V}$ can calculate all intermediate results as follows:
\begin{equation*}
  {z}_j = f_j({z}_{j-1}), \; \text{for each $j \in \{1,\dots,n\}$}
\end{equation*}

Of course, we additionally set ${z}_0 := {x}$. If everything was computed
correctly and the function was split into shards correctly, eventually, we will
have ${z}_n = {y}$. We will give a practical example of this in the further
sections.

So recall that $\mathcal{P}$ (referred to in \textit{BitVM2} as the
\textit{operator})
only needs to prove that the given program $f$ indeed returns ${y}$ for \({x}\),
otherwise \textbf{anyone can disprove this fact}. In our case, this means giving
challengers (essentially, being verifiers $\mathcal{V}$) the ability to prove
that at least one of the sub-program statements \(f_j({z}_{j-1}) = {z}_j\) is
false.

So overall, the idea of \textit{BitVM2} can be described as follows:
\begin{enumerate}
  \item The program $f$ is decomposed into shards $f_1,\dots,f_n$ of
    reasonable size\footnote{By ``size'' we mean the number of
    \texttt{OP\_CODES} needed to represent the logic.}.
  \item $\mathcal{P}$ executes $f$ on input ${x}$ shard by shard, obtaining
    intermediate steps ${z}_1,\dots,{z}_n$.
  \item $\mathcal{P}$ commits to the given intermediate steps and publishes
    commitments on-chain.
  \item $\mathcal{V}$, knowing ${x}$ published by $\mathcal{P}$, executes the
    same program, obtaining his own states
    $\widetilde{z}_1,\dots,\widetilde{z}_n$.
  \item $\mathcal{V}$ checks whether $\widetilde{z}_j = z_j$. If this does not
    hold, the verifier publishes transactions corresponding to the disprove
    statement $z_j \neq f_j(z_{j-1})$ and claims funds.
\end{enumerate}

\subsection{Implementation on Bitcoin}

This does not sound very hard; however, implementing this in Bitcoin is not
obvious. The good news is that Bitcoin is a stack-based language, so the
function $f$ is just a string, where each word is the \texttt{OP\_CODE}. Notice
that, in the stack-based languages, the concatenation $f_1 \parallel f_2$ of two
\textit{valid} functions $f_1$ and $f_2$ is the same thing as their composition.
In other words, executing the script \script{\elem{x} \elem{f_1} \elem{f_2}} is
the same as calculating composition $f_2 \circ f_1(x)$. So all what remains is
finding \textit{valid} $f_1,\dots,f_n$ such that $f = f_1 \parallel f_{2}
\parallel \dots \parallel f_n$. All the intermediate steps ${\{z_j\}}_{0 \leq j
\leq n}$ can be calculated as specified in \Cref{alg:intermediate_steps}.

\begin{algorithm}[H]
  \caption{Calculating intermediate steps from script shard decomposition}
  \Input{Script $f$}
  \Output{Intermediate steps $z_1,\dots,z_n$}

  Decompose $f$ into shards: $(f_1,\dots,f_n) \gets \mathsf{Decompose}(f)$;

  \For{$i \in \{1,\dots,n\}$}{
    $z_i \gets \mathsf{Exec}(\{\elem{z_{i-1}} \elem{f_i}\})$;
  }

  \Return{$z_1,\dots,z_n$}
\end{algorithm}\label{alg:intermediate_steps}

\begin{example}
  Consider a fairly simple program $f$:
  \begin{equation*}
    f(a,b) = 25a^2b^2{(a+b)}^2
  \end{equation*}

  Additionally, assume that we can abstract the multiplication operation
  via the opcode \texttt{OP\_MUL} (which, in fact, is already natively
    implemented in Bitcoin Script, although the size of such script
  for two 254-bit numbers exceeds 60kB).
  Then, the implementation of $f$ in Bitcoin Script could be:
  \begin{empheqboxed}
    \begin{align*}
      &\elem{a} \elem{b} \opcode{\texttt{OP\_2DUP}}
      \opcode{\texttt{OP\_ADD}} \opcode{\texttt{OP\_MUL}}
      \opcode{\texttt{OP\_MUL}}
      \opcode{\texttt{OP\_DUP}} \opcode{\texttt{OP\_DUP}} \\&
      \opcode{\texttt{OP\_ADD}} \opcode{\texttt{OP\_DUP}}
      \opcode{\texttt{OP\_ADD}} \opcode{\texttt{OP\_ADD}}
      \opcode{\texttt{OP\_DUP}} \opcode{\texttt{OP\_MUL}}
    \end{align*}
  \end{empheqboxed}

  Fairly complex, right? Let us split the function into three shards
  $\textcolor{red!80!white}{f_1}$, $\textcolor{blue}{f_2}$, and
  $\textcolor{green!60!black}{f_3}$:
  \begin{align*}
    \textcolor{red!80!white}{f_1}(x,y) = xy(x+y), \quad
    \textcolor{blue}{f_2}(z) = 5z, \quad
    \textcolor{green!60!black}{f_3}(w) = w^2
  \end{align*}

  This way, it is fairly easy to see that $f(a,b) =
  \textcolor{green!60!black}{f_3} \circ \textcolor{blue}{f_2} \circ
  \textcolor{red!80!white}{f_1}(a,b)$. In turn, in Bitcoin
  script we can represent $f$ as $\textcolor{red!80!white}{f_1}
  \parallel \textcolor{blue}{f_2} \parallel \textcolor{green!60!black}{f_3}$:
  \begin{empheqboxed}
    \begin{align*}
      &\elem{a} \elem{b}
      \textcolor{red!80!white}{\opcode{\texttt{OP\_2DUP}}
        \opcode{\texttt{OP\_ADD}} \opcode{\texttt{OP\_MUL}}
      \opcode{\texttt{OP\_MUL}}} &&
      \textcolor{gray!80!black}{\text{// $xy(x+y)$}} \\
      &\textcolor{blue}{\opcode{\texttt{OP\_DUP}}
        \opcode{\texttt{OP\_DUP}} \opcode{\texttt{OP\_ADD}}
        \opcode{\texttt{OP\_DUP}} \opcode{\texttt{OP\_ADD}}
      \opcode{\texttt{OP\_ADD}}} && \textcolor{gray!80!black}{\text{// $5z$}} \\
      &\textcolor{green!60!black}{\opcode{\texttt{OP\_DUP}}
      \opcode{\texttt{OP\_MUL}}}&&\textcolor{gray!80!black}{\text{// $w^2$}}
    \end{align*}
  \end{empheqboxed}
\end{example}

Bad news is that $\mathsf{Decompose}$ function is quite tricky to implement.
Namely, we believe that there are several issues:
\begin{itemize}
  \item Decomposition must be valid, meaning each $f_j$ must be valid itself.
    For example, $f_j$ cannot contain unclosed \texttt{OP\_IF}'s. This issue is
    easily fixed through a careful implementation of the splitting mechanism:
    for instance, whenever the number of \texttt{OP\_IF}'s and
    $\texttt{OP\_NOTIF}$'s is not equal to the number of \texttt{OP\_ENDIF}'s,
    we continue the current shard until the balance is restored.
  \item Despite that each $f_j$ might be small, not necessarily $z_j$ is. In
    other words, optimizing the size of each $f_j$ does not result in optimizing
    the size of $z_j$. Moreover, in the further sections, we show that
    optimizing the size of intermediate states is, in fact, a much more tricky
    and fundamental issue than optimizing the shards' sizes. In other words, we
    should find a balance between the size of $f_j$ and the size of $z_j$.
  \item Some of $z_j$'s might contain the same repetitive pieces: for example,
    the lookup table for certain algorithms or the number binary/$w$-width
    decomposition for arithmetic. We believe that there must be an optimal
    method to store commitments.
\end{itemize}

However, the default version proceeds as follows: suppose our script is of form
$f = \{ \elem{s_1} \elem{s_2} \ldots \elem{s_k} \}$ where $\elem{s_j}$ is either
an \texttt{OP\_CODE} or an element in the stack (added via, for example,
\texttt{OP\_PUSHBYTES}). Then, we start splitting the program from left to right
and if the size of the current shard exceeds the limit (say, $L$), we stop and
start a new shard. The only exception when we cannot stop is unclosed
\texttt{OP\_IF} and \texttt{OP\_NOTIF}. This way, approximately, we will have
$\lceil k/L \rceil$ shards each of size $L$.

\subsubsection{Fuzzy Search}

The basic version, though, does not guarantee the optimal intermediate stack
sizes. One of the proposals to improve the splitting mechanism is to make
program automatically choose the optimal size. In other words, we make the
parameter $L$ variable and try to find the optimal $L$ that minimizes the
certain ``metric''. What is this metric?

Since we want to potentially disprove the equality $z_{j+1} = f_j(z_j)$, the
cost of such disproof is the total size of $z_j$, $z_{j+1}$ and the shard $f_j$.
Denote the size of the script/state by $|\star|$. Then, we want to minimize some
sort of ``average'' of $\alpha(|z_j| + |z_{j+1}|) + |f_j|$. The factor $\alpha$
is introduced since, besides the cost of storing $z_j$, we also need to
\textit{commit} to these values which, as we will see, significantly increases
the cost of a disprove script. In other words, $\alpha$ is a considerable factor
in practice: currently, our estimate suggests $\alpha \approx 1000$.

Then, depending on the goal, we might choose different criteria of
``averaging'':
\begin{itemize}
  \item \textbf{Maximal size}. Suppose we want to minimize the cost of the
    worst-case scenario. Suppose after the launching the splitting mechanism on
    the shard size $L$ we get $k_L$ shards $f_{L,1},\dots,f_{L,k_L}$ with
    intermediate states $z_{L,0},\dots,z_{L,k_L}$. Then, we choose $L$ to be:
    \begin{equation*}
      \hat{L} := \argmin_{0 \leq L \leq L_{\max}} \left\{ \max_{ 0
        \leq j < k_L } \left\{ \alpha(|z_{L,j}| + |z_{L,j+1}|) +
      |f_{L,j}| \right\} \right\}\;.
    \end{equation*}
  \item \textbf{Average size}. Suppose we want to minimize the average cost of
    disproof. Then, we choose $L$ to be:
    \begin{equation*}
      \hat{L} := \argmin_{0 \leq L \leq L_{\max}} \left\{
        \frac{1}{k_L} \sum_{0 \leq j < k_L} \left( \alpha(|z_{L,j}| +
      |z_{L,j+1}|) + |f_{L,j}| \right) \right\}\;.
    \end{equation*}
\end{itemize}

Note, however, that this algorithm is still far from being the most optimal one.
We assume that, in reality, in the majority of cases, the optimal shards sizes
can significantly differ, which the automatic splitting can easily miss.

The ultimate solution would be to check every possible splitting and choose the
one that minimizes the cost of disproof. However, this is not feasible in
practice, as the number of possible splittings is enormous (even, say, for the
fixed number of shards).

\subsubsection{Current State}
We implemented the basic splitting mechanism that finds $f_1,\dots,f_k$ of
almost equal size (which can be specified). It already produces valid shards and
intermediate states on all of the following scripts:
\begin{itemize}
  \item \textbf{Big Integer Addition} (of any bitsize).
  \item \textbf{Big Integer Multiplication} (of any bitsize).
  \item \textbf{SHA-256} hash function.
  \item \textbf{Square Fibonacci Sequence Demo}.
  \item \textbf{\texttt{u32} Multiplication}.
\end{itemize}

We will explore the last two functions in more detail a bit later. All the
current implementation of test scripts can be found through the link below:
\begin{center}
  \url{https://github.com/distributed-lab/bitvm2-splitter/tree/main/bitcoin-testscripts}
\end{center}

\section{Assert Transaction}\label{sec:assert-tx}

When the splitting is ready, the prover $\mathcal{P}$ publishes an
\texttt{Assert} transaction, which has one output with multiple
possible spending scenarios:

\begin{enumerate}
  \item \texttt{PayoutScript} (\texttt{CheckSig} +
    \texttt{CheckLocktimeVerify} + \texttt{Covenant}) --- the transaction
    has passed verification, and the operator can spend the output,
    thereby confirming the statement $y=f(x)$.
  \item $\texttt{DisproveScript[\text{$i$}]}$ --- one of the challengers
    has found a discrepancy in the intermediate states \(z_i\),
    \(z_{i-1}\) and the sub-program \(f_i\). In other words, they have
    proven that \(f_i(z_{i-1}) \neq z_i\), and thus, they can spend the
    output.
\end{enumerate}

While the \texttt{PayoutScript} is rather trivial, we need to specify how the
\texttt{DisproveScript[\text{$i$}]} is constructed.\ \texttt{DispoveScript} is
part of the MAST tree in a Taproot address, allowing the verifier to claim the
transaction amount for states \(z_i\), \(z_{i-1}\), and sub-program \(f_i\). We
call it \(\texttt{DisproveScript[\text{$i$}]}\) and compose it as follows:

\begin{empheqboxed}
  \begin{align*}
    \elem{z_i} \elem{z_{i-1}} \elem{f_i} \opcode{OP\_EQUAL} \opcode{OP\_NOT}
  \end{align*}
\end{empheqboxed}

This script does not need a \texttt{CheckSig}, as with the correct \(z_i\) and
\(z_{i-1}\), it will consistently execute successfully. Therefore, we added a
Winternitz signature and covenant verification to restrict the script's spending
capability. Currently, we will simulate covenant through a committee of a single
person (essentially, being a single signature verification), but this is easily
extendable to the multi-threshold signature version (and, potentially, to
\texttt{OP\_CAT}-based version, but that is the next phase of our research).

\subsection{Winternitz Signature}\label{sec:lamport-signature}

Unlike other digital signature algorithms, the Winternitz signature uses a pair
of random secret and public keys $(\mathsf{sk}, \mathsf{pk})$ that can sign and
verify only any message from the message space \(\mathcal{M} = {\{0,
1\}}^{\ell}\) of $\ell$-bit messages.

However, once the signature $\sigma_{m}$ is formed, where $m \in \mathcal{M}$ is
the message being signed, \((\mathsf{sk}_{m}, \mathsf{pk}_{m})\) become tied to
\(m\), because any other signature with these keys will compromise the keys
themselves. Thus, for the message \(m\), the keys \((\mathsf{sk}_{m},
\mathsf{pk}_{m})\) are one-time use.

Now, let us define the Winternitz Signature. Further by $f^{(k)}(x)$ denote the
composition of function $f$ with itself $k$ times: $f^{(k)}(x) = \underbrace{f
\circ \dots \circ f}_{k \; \text{times}}(x)$.

\begin{definition}
  The \textbf{Winternitz Signature Scheme} over parameters $(k,d)$ with a hash
  function $H: \mathcal{X} \to \mathcal{X}$ is defined as follows:
  \begin{itemize}
    \item $\textsc{Gen}(1^{\lambda})$: secret key is generated as a tuple
      $(x_1,\dots,x_k) \xleftarrow{R} \mathcal{X}$, while the public key is
      $(y_1,\dots,y_k)$, where $y_j = H^{(d)}(x_j)$ for each $j \in
      \{1,\dots,k\}$.
    \item $\textsc{Sign}(m,\mathsf{sk})$: denote by $\mathcal{I}_{d,k} :=
      {(\{0,\dots,d\})}^k$ and suppose we have an encoding function
      $\mathsf{Enc}: \mathcal{M} \to \mathcal{I}_{d,k}$ that translates a
      message $m \in \mathcal{M} = {\{0,1\}}^{\ell}$ to the element in space
      $\mathcal{I}_{d,k}$. Now, set $e = (e_1,\dots,e_k) \gets \mathsf{Enc}(m)$.
      Then, the signature is formed as:
      \begin{equation*}
        \sigma \gets ({H}^{(e_1)}(x_1), H^{(e_2)}(x_2), \dots, H^{(e_k)}(x_k))
      \end{equation*}
    \item $\textsc{Verify}(\sigma,m,\mathsf{pk})$: to verify $\sigma =
      (\sigma_1,\dots,\sigma_k)$ on $m \in \mathcal{M}$ and
      $\mathsf{pk}=(y_1,\dots,y_k)$, first compute encoding $(e_1,\dots,e_k)
      \gets \mathsf{Enc}(m)$ and then check whether:
      \begin{equation*}
        H^{(d-e_j)}(\sigma_j) = y_j, \quad j \in \{1,\dots,k\}.
      \end{equation*}
  \end{itemize}
\end{definition}

That being said, by taking the intermediate states ${\{z_j\}}_{1 \leq j \leq n}$
as the message for the Winternitz signature, we form one-time key pairs
${\{(\mathsf{sk}_j,\mathsf{pk}_j)\}}_{1 \leq j \leq n}$ and signatures
${\{\sigma_j\}}_{1 \leq j \leq n}$, respectively (where each of $\mathsf{pk}_j$,
$\mathsf{sk}_j$, and $\sigma_j$ corresponds to the intermediate variable $z_j$).
Then, to spend the output from the \texttt{Assert} transaction using the
$\texttt{DisproveScript[\text{$j$}]}$ script, the challenger is required to add
the corresponding states $z_j$, $z_{j-1}$, and corresponding signatures
$\sigma_j$, $\sigma_{j-1}$ to the stack in the \texttt{scriptSig}, making the
\texttt{scriptSig} of the transaction input like this:

\begin{empheqboxed}
  \begin{align*}
    &\elem{z_{j-1}} \opcode{OP\_DUP} \elem{\sigma_{j-1}}
    \elem{\mathsf{pk}_{j-1}} \opcode{OP\_WINTERNITZVERIFY} \\
    &\elem{z_{j}} \opcode{OP\_DUP} \elem{\sigma_{j}}
    \elem{\mathsf{pk}_{j}} \opcode{OP\_WINTERNITZVERIFY} \\
    &\elem{f_j} \opcode{OP\_EQUAL} \opcode{OP\_NOT}
  \end{align*}
\end{empheqboxed}

where \texttt{OP\_WINTERNITZVERIFY} is the verification of the
Winternitz signature (commitment), described in Bitcoin Script (as
Bitcoin Script does not have a built-in \texttt{OP\_CODE} for
Winternitz signatures)\footnote{Its implementation can be found here:
  \url{https://github.com/distributed-lab/bitvm2-splitter/blob/feature/winternitz/bitcoin-winternitz/src/lib.rs}.}.

\subsubsection{Winternitz Signatures in Bitcoin
Script}\label{sec:winternitz-in-bitcoin-script}

The first biggest issue with the provided approach is that the Winternitz Script
requires encoding the message $\mathsf{Enc}(m)$, which splits the state into $d$
digit number. For \textit{BitVM2}, it means encoding each state
$z_j$. However, the
arithmetic in Bitcoin Script is limited and contains only basic opcodes such as
\texttt{OP\_ADD}. To make matters worse, all the corresponding operations can be
applied to 32-bit elements only, and as the last one is reserved for a sign,
only 31 bits can be used to store the state. This limitation can be considered
strong, but most of the math can be implemented through 32-bit stack elements.
So lets fix $\ell = 32$ --- maximum size of the stack element in bits.

The first observation is that essentially $z_j$ is a collection of
32-bit numbers (suppose this collection consists of $n_j$ numbers).
Denote this fact by $z_j = (u_{j,1}, u_{j,2}, \dots, u_{j, n_j})$
where each $u_{j,k} \in \mathbb{Z}_{2^{\ell}}$. Therefore, one way to
implement the message encoding is following:
\begin{enumerate}
  \item Aggregate elements of $z_j$ into a single hash digest $h_j \gets
    H(u_{j,1} \parallel u_{j,2} \parallel \dots \parallel u_{j,n_j})$.
  \item Use dominant free function $P(h_j)$ as described
    in~\cite{applied-crypto} to get the decomposition.
\end{enumerate}

However, as of now, the Bitcoin does not have the \texttt{OP\_CAT}, so there is
no way we can effectively aggregate the intermediate state $z_j$ into a single
stack element. Meaning, we need to create a Winternitz keypair
$(\mathsf{pk}_{j,k}, \mathsf{sk}_{j,k})$ for each $u_{j,k}$ where $k \in
\{1,\dots,n_j\}$.

However, there are couple of tricks to make the life easier. First, obviously,
it is convenient to choose $d$ such that $d+1=2^w$ for some $w \in \mathbb{N}$.
This splits the signed message $m$ by the fixed number of equal chunks of $w$
bits. Let $N$ be the sum of $n_0$ --- the number of $d$-digit numbers in the
decomposition of $m$, and $n_1$ a checksum (see \Cref{tab:winternitz}).

\iffalse{}
%The python script i used for this table:
\begin{verbatim}
import math
l = 32
ds = [3, 7, 15, 31, 63, 127, 255]

for d in ds:
    w = math.ceil(math.log(d+1, 2))
    n0 = math.ceil(l / w)
    n1 = math.ceil((2**w * n0).bit_length() / w)
    k = n0 + n1
    print(f"{w} & {n0} & {n1} & {k} \\\\")
\end{verbatim}
\fi

\begin{table}[h]
  \centering
  \begin{tabular}{lcccc}
    \toprule
    $w$ & $n_0$ & $n_1$ & $N$ \\
    \midrule
    2 & 16 & 4 & 20 \\
    3 & 11 & 3 & 14 \\
    4 & 8 & 2 & 10 \\
    5 & 7 & 2 & 9 \\
    6 & 6 & 2 & 8 \\
    7 & 5 & 2 & 7 \\
    8 & 4 & 2 & 6 \\
    \bottomrule
  \end{tabular}
  \caption{Different values of $N$ depending on $d$ for 32-bit
  message}\label{tab:winternitz}
\end{table}

Secondly, notice the following fact: \emph{encoding the message $m$
  (essentially, being the number decomposition) is more expensive than decoding
the message (being the number recovery from limbs)}. In fact, if chunks are of
equal lengths, the recovery of a message $m$ from $n_0$ digits can be computed
very easily: simply set $m \gets \sum_{i=0}^{n_0} e_i \times 2^{wi}$. Note that
multiplication by powers of two can be implemented in Bitcoin Script with
sequence of \texttt{OP\_DUP} and \texttt{OP\_ADD} opcodes quite efficiently. So,
for example, multiplication of $e_j$ by $2^n$ in Bitcoin Script is:

\begin{empheqboxed}
  \begin{align*}
    \elem{e_j} \underbrace{\opcode{OP\_DUP} \opcode{OP\_ADD}}_{n \;
    \text{times}}
  \end{align*}
\end{empheqboxed}

As Bitcoin Script has no loops or jumps, implementing dynamic number of
operations, like hashing something $d - e_j$ times without knowing the $e_j$
beforehand is challenging. That's why implementation uses the ``lookup'' table
of all $d$ hashes of signature's part $\sigma_j$ and by using \texttt{OP\_PICK}
pop the $d - e_j$ one on the top of stack, like this:

\begin{empheqboxed}
  \begin{align*}
    & \elem{\sigma_j} \underbrace{\opcode{OP\_DUP}
    \opcode{OP\_HASH}}_{d \; \text{times}} \\
    & \elem{e_j} \opcode{OP\_PICK}
  \end{align*}
\end{empheqboxed}

So overall, the algorithm to sign the states looks as follows:
\begin{enumerate}
  \item The prover $\mathcal{P}$ runs the program on all shards
    ${\{f_j\}}_{1 \leq
    j \leq n}$ to obtain the intermediate states ${\{z_j\}}_{0 \leq j \leq n}$:
    essentially, being the stack after executing \script{\elem{z_{j-1}}
    \elem{f_j}}.
  \item $\mathcal{P}$ interprets each $z_j$ as a collection of $n_j$ 32-bit
    numbers: $z_j = (u_{j,1}, u_{j,2}, \dots, u_{j, n_j})$.
  \item $\mathcal{P}$ encodes each $u_{j,k}$ and forms the Winternitz keypairs
    ${\{(\mathsf{pk}_{j,k},\mathsf{sk}_{j,k})\}}_{1 \leq k \leq n_j}$.
  \item Verifier $\mathcal{V}$, when publishing the
    \texttt{DisproveScript[\text{$j$}]}, will add the corresponding
    \textbf{encoded} states $\mathsf{Enc}(z_j)$, $\mathsf{Enc}(z_{j-1})$, and
    corresponding signatures $\sigma_j$ and $\sigma_{j-1}$.
  \item The script, in turn, besides the verification of the intermediate states
    signatures, will \textbf{recover} the original $u_{j,k}$ elements from the
    encoded states and verify the equality $f_j(z_{j-1}) = z_j$ after
    recovery of
    both $z_j$ and $z_{j-1}$.
\end{enumerate}

\textbf{Script Size Analysis.} Still, even with optimizations
provided, the current implementation requires around 1000 bytes per
32-bit stack element, which is unfortunatly a lot. Parts of the public
key make the largest contribution to the script size. Assuming that as
$H$ implementation uses \texttt{OP\_HASH160}, each part
$(y_1,\dots,y_N)$ of the public key $\mathsf{pk}_{m}$ adds $n_H := 20$ bytes to
the total script size. Additionaly, for calculating a lookup table for
signature verification, $2d \times N$ opcodes are used. Furthermore, for
message recovery, $2\sum_{i = 0}^{n_0} i w$ opcodes are added too. Also,
note that $2 \sum_{i = 0}^{n_0} i w = w n_0 (n_0+1) \approx w n_0^2$, so the
total script size, excluding utility opcodes, will be at least roughly
$n_H N + 2 dN + w n_0^2$.

For script's input stack elements, the size of encoding
$\mathsf{Enc}(z_j)$ and corresponding signature $\sigma_j$, as each limb
can be stored only as a whole byte, is around $N + n_H N = (1+n_H)N$
bytes. So the total size of the largest part of a disprove transaction
--- \texttt{witness}, is roughly $2N(n_H + d) + wn_0^2$. The specific sizes
for different $d$ can be seen in \Cref{tab:winternitz-script-size}.

\iffalse{}
%The python script i used for this table:
\begin{verbatim}
import math
l = 32
ds = [3, 7, 15, 31, 63, 127, 255]

for d in ds:
    w = math.ceil(math.log(d+1, 2))
    n0 = math.ceil(l / w)
    n1 = math.ceil((2**w * n0).bit_length() / w)
    k = n0 + n1

    pk_size = 20 * k
    ver_size = 2 * d * k
    sig_size = 21*k
    rec_size = 0
    for i in range(0, n0):
        rec_size += int(i * w)

    total = pk_size + ver_size + rec_size + sig_size

    print(f"{d} & {sig_size} & {pk_size} & {ver_size} & {rec_size} & {total} \\\\")
\end{verbatim}
\fi
\begin{table}[H]
  \centering
  \begin{tabular}{cccccc}
    \toprule
    $d$ & \textbf{Signature} & \textbf{Public Key} &
    \textbf{Verification Script} & \textbf{Recovery Script} & \textbf{Total} \\
    \midrule
    3 & 420 & 400 & 120 & 240 & 1180 \\
    7 & 294 & 280 & 196 & 165 & 935 \\
    15 & 210 & 200 & 300 & 112 & 822 \\
    31 & 189 & 180 & 558 & 105 & 1032 \\
    63 & 168 & 160 & 1008 & 90 & 1426 \\
    127 & 147 & 140 & 1778 & 70 & 2135 \\
    255 & 126 & 120 & 3060 & 48 & 3354 \\
    \bottomrule
  \end{tabular}
  \caption{Different script sizes depending on the $d$ value per each
    32-bit message. Note that, ``Signature'' column includes the
  encoding of 32-bit message in it.}\label{tab:winternitz-script-size}
\end{table}

\subsubsection{Compact Winternitz commitment
scheme in Bitcoin}\label{sec:compact-winternitz-in-bitcoin}

As we mentioned in prevoius section, the most contribution to script
size comes from public key, signature, verification and recovery, but
do we actually need to sign each integer limb?

Recall that the checksum of the Winternitz singature described
in~\cite{applied-crypto} is $c = d n_0 - \sum_{i=1}^{n_0}e_i$. Suppose, at some
point during the protocol execution with $d=15$, the last limb of the message
encoding $\mathsf{Enc}(m)$ turns out to be $e_{7} = 0$. According to
checksum expression, the value of $e_{7}$ does not contribute to the
sum of the checksum, so, without any security concerns, can be omitted. But to
keep the right value after a recover, we can't remove all zero limbs. That's why
instead, we propose \textit{skipping} only the most significant \textit{zero}
limbs of a stack element.

This makes the recovery script size equal to zero in the best-case
scenario. The public key and signature hash digest number ranges from
$1 + n_1$ in the best case to $N$ in the worst. The same applies to
the total verification script size, as fewer checks are required for
smaller public keys.

\iffalse{}
%The python script i used for this table:
\begin{verbatim}
import math
l = 32
d = 15
w = math.ceil(math.log(d+1, 2))
n0 = math.ceil(l / w)
n1 = math.ceil((2**w * n0).bit_length() / w)

for z in range(1, 8+1):
    k = z + n1

    pk_size = 20 * k
    ver_size = 2 * d * k
    sig_size = 21*k
    rec_size = 0
    for i in range(0, z):
        rec_size += int(i * w)

    total = pk_size + ver_size + rec_size + sig_size

    print(f"{z} & {sig_size} & {pk_size} & {ver_size} & {rec_size} & {total} \\\\")
\end{verbatim}
\fi
\begin{table}[H]
  \centering
  \begin{tabular}{cccccc}
    \toprule
    \textbf{Non-zero limbs} & \textbf{Signature} & \textbf{Public
    Key} & \textbf{Verification} & \textbf{Recovery} & \textbf{Total} \\
    \midrule
    1 & 63 & 60 & 90 & 0 & 213 \\
    2 & 84 & 80 & 120 & 4 & 288 \\
    3 & 105 & 100 & 150 & 12 & 367 \\
    4 & 126 & 120 & 180 & 24 & 450 \\
    5 & 147 & 140 & 210 & 40 & 537 \\
    6 & 168 & 160 & 240 & 60 & 628 \\
    7 & 189 & 180 & 270 & 84 & 723 \\
    8 & 210 & 200 & 300 & 112 & 822 \\
    \bottomrule
  \end{tabular}
  \caption{Different number of possible script sizes for $d = 15$
    depending on the the number of non-zero
  limbs.}\label{tab:winternitz-script-size}
\end{table}

\subsection{Disprove Script Specification}

All in all, the $\texttt{DisproveScript[\text{$j$}]}$ is formed as follows:
\begin{itemize}
  \item \textbf{Witness:} $\Big\{\mathsf{Enc}(z_{j+1})$,
    $\sigma_{j+1}$, $\mathsf{Enc}(z_{j})$, $\sigma_j\Big\}$.
  \item \textbf{Spending Condition}:
    \begin{empheqboxed}
      \begin{align*}
        &\elem{\mathsf{pk}_j} \opcode{\texttt{OP\_WINTERNITZVERIFY} }
        \opcode{\texttt{OP\_RESTORE}} \opcode{\texttt{OP\_TOALTSTACK}} \\
        &\elem{\mathsf{pk}_{j}} \opcode{\texttt{OP\_WINTERNITZVERIFY}
        } \opcode{\texttt{OP\_RESTORE}} \opcode{\texttt{FROM\_TOALTSTACK}} \\
        &\elem{f_j} \opcode{ \texttt{OP\_EQUAL} } \opcode{\texttt{OP\_NOT}}
      \end{align*}
    \end{empheqboxed}
\end{itemize}

\textbf{Note on implementation.} One more tricky part is that $z_j$, in fact, is
not really a collection of stack elements, but two collections: one sitting in
the \texttt{mainstack}, while the other is in the \texttt{altstack}. For that
reason, when signing $z_j$, what we \textit{really} mean is signing both
elements in the \texttt{mainstack} and the \texttt{altstack}. Finally, one
should carefully manage when to pop the elements in and out of the
\texttt{altstack}. All of this is implemented in the current version of the
code.

\subsection{Structure of the MAST Tree in a Taproot
Address}\label{sec:mast-tree-structure}

The inputs of the \texttt{Assert} transaction spend the output to a Taproot
address, which consists of a MAST tree of Bitcoin scripts mentioned in
\Cref{sec:assert-tx}. From the \textit{BitVM2} document, it is known
that the first \(n\)
scripts in the tree are all \(\texttt{DisproveScript[\text{$i$}]}\), where \(i
\in \{1,\dots, n\}\), and the last is a script that allows the operator who
published the \texttt{Assert} transaction to spend the output after some time. A
visualization of this tree can be seen in the \Cref{fig:assert-tx-mast-tree}.

% Drawing the Figure
\tikzset{
  leaf/.style={rectangle, draw=green!90, fill=green!10, very thick,
  rounded corners},
  root/.style={rectangle, draw=gray!80, fill=gray!20, very thick,
  rounded corners},
  script/.style={rectangle, draw=blue!80, fill=blue!10, very thick,
  text width=2.5cm, minimum height=1cm, align=center},
  arrow/.style={thick,-{Stealth[round]},shorten >=2pt,shorten <=2pt}
}

\begin{figure}[h]
  \centering
  \begin{tikzpicture}[node distance=1cm and 2cm]

    % Root node
    \node (root) [root] {\textsf{Root}};

    % Write a text left to root
    \node [left=of root, xshift=2.1cm] {$\mathsf{0x000\dots} + G \times$};

    % Leaf nodes
    \node (leaf1) [leaf, below left=of root] {$\mathsf{Leaf}_1$};
    \node (leaf2) [leaf, below right=of root] {$\mathsf{Leaf}_2$};

    % Script nodes
    \node (script1) [script, below=of leaf1, xshift=-1.5cm]
    {\scriptsize\texttt{DisproveScript[\text{$1$}]}};
    \node (script2) [script, below=of leaf1, xshift=1.5cm]
    {\scriptsize\texttt{DisproveScript[\text{$2$}]}};
    \node (script3) [script, below=of leaf2, xshift=-1.5cm]
    {\scriptsize\texttt{DisproveScript[\text{$3$}]}};
    \node (script4) [script, below=of leaf2, xshift=1.5cm]
    {\scriptsize\texttt{PayoutScript}};

    % Draw arrows
    \draw[arrow] (root) -- (leaf1);
    \draw[arrow] (root) -- (leaf2);
    \draw[arrow] (leaf1) -- (script1);
    \draw[arrow] (leaf1) -- (script2);
    \draw[arrow] (leaf2) -- (script3);
    \draw[arrow] (leaf2) -- (script4);

  \end{tikzpicture}
  \caption{\label{fig:assert-tx-mast-tree}Script tree in a Taproot
    address with three sub-programs and two intermediate states. Here,
  $G$ is the generator point of the elliptic curve.}
\end{figure}

\section{Transactions Graph}\label{sec:txs-graph}

To implement whole transaction flow as cheap as possible in most
optimistic cases, BitVM2 proposes a graph of transactions which
introduces multiple paths of interaction between prover (operator) and
verifier (comittee):

\begin{itemize}
\item \texttt{Claim} --- a transaction with the initial statement
  $f(x) = y$ assertion without any commitments to intermidiate states.
\item \texttt{OptimisticPayout} --- a transaction that, without any
  dispute, prover $\mathcal{P}$ uses to spend the asserted in \texttt{Claim}
  amount.
\item \texttt{Challenge} --- after publishing the \texttt{Challenge}
  transactions, the \texttt{OptimisticPayout} dispute resolving is
  blocked, so operator \textbf{MUST} publish the \texttt{Assert}
  transaction.
\item \texttt{Assert} --- the transaction that publishes all commitments
  $\sigma_1, \ldots, \sigma_n$ to intermidiate states
  $z_1, \ldots, z_n$ and opens the ability for verifier $\mathcal{V}$ to punish the
  prover $\mathcal{P}$ if program execution was invalid.
\item \texttt{Disprove} --- a transaction that spends \texttt{Assert}
  one if one of the program chunks published by verifier $\mathcal{V}$ invalid.
\item \texttt{Payout} --- a transaction that returns from
  \texttt{Assert} one staked amount back to prover $\mathcal{P}$ and is
  spendable after some time if verifier $\mathcal{V}$ can't find an invalid
  program chunk.
\end{itemize}

To emulate covenants out implementations uses interactive
multisignature Schnorr based scheme Musig2~\cite{musig2}. Highlevel
view of the transactions graph can be seen in the
Figure~\ref{fig:txs-graph}. But we'll disscuss each transaction in
detail in next sections.

\tikzset{
  claimtx/.pic = {
    % Draw the transaction box
    \node[rounded corners, minimum width=3.5cm, minimum height=2.5cm, fill=gray!30, thick, draw=gray]
    (cltxbody) at (0,0) {};
    % Draw the label of tx
    \node[above] at (cltxbody.north) {\texttt{Claim}};
    % Draw the input
    \node[rounded corners, minimum width=2cm, minimum height=1cm, fill=green!30, thick, draw=green]
    (input) at (-1.2,0) {$*, x$ ($d$ \bitcoin)};
    \node[left] at (input.west) {$\mathcal{P}$};
    % Draw the outputs
    \node[rounded corners, minimum width=2cm, minimum height=1cm, fill=blue!30, thick, draw=blue]
    (cloutput1) at (1.2,0.2) {$d$ \bitcoin};
    \node[rounded corners, minimum width=2cm, minimum height=1cm, fill=blue!30, thick, draw=blue]
    (cloutput2) at (1.2,-0.9) {0.00001 \bitcoin};
  },
  challengetx/.pic = {
    % Draw the transaction box
    \node[rounded corners, minimum width=3.5cm, minimum height=2cm, fill=gray!30, thick, draw=gray]
    (chtxbody) at (0,0) {};
    % Draw the label of tx
    \node[above] at (chtxbody.north) {\texttt{Challenge}};
    % Draw the input
    \node[rounded corners, minimum width=2cm, minimum height=1cm, fill=green!30, thick, draw=green]
    (chinput1) at (-1.5,0.2) {$\sigma_{\mathcal{P}}$ (0.00001 \bitcoin)};
    \node[rounded corners, minimum width=2cm, minimum height=1cm, fill=green!30, thick, draw=green]
    (input) at (-1.5,-1) {$*$ ($c$ \bitcoin)};
    % show that this input is crowdfunded
    \node[left] at (input.west) {\texttt{Crowdfunded}};
    % Draw the outputs
    \node[rounded corners, minimum width=2cm, minimum height=1cm, fill=blue!30]
    (output) at (1.5,0.2) {$c$ \bitcoin};
    \node[right] at ($(output.east) + (0.2,0)$) {$\mathcal{P}$};

    \draw[dashed, thick, rounded corners] (-2.8,-0.3) rectangle (2.5,0.7);
    % label that this is SINGLE|ANYONECANPAY
    \node[below] at (2.6,-0.35) {\texttt{SINGLE|ANYONECANPAY}};
  },
  payoutoptimistictx/.pic = {
    % Draw the transaction box
    \node[rounded corners, minimum width=3.5cm, minimum height=2cm, fill=gray!30, thick, draw=gray]
    (optxbody) at (0,0) {};
    % Draw the label of tx
    \node[above] at (optxbody.north) {\texttt{OptimisticPayout}};
    % Draw the input
    \node[rounded corners, minimum width=2cm, minimum height=1cm, fill=green!30, thick, draw=green]
    (opinput1) at (-1.5,0.2) {$\sigma_{\mathcal{V} + \mathcal{P}}$ ($d$ \bitcoin)};
    \node[rounded corners, minimum width=2cm, minimum height=1cm, fill=green!30, thick, draw=green]
    (opinput2) at (-1.5,-0.9) {$\sigma_{\mathcal{P}}$ (0.00001 \bitcoin)};
    % Draw output
    \node[rounded corners, minimum width=2cm, minimum height=1cm, fill=blue!30, thick, draw=blue]
    (opoutput1) at (1.5,0.2) {$d$ \bitcoin};
    \node[right] at (opoutput1.east) {$\mathcal{P}$};
  },
  asserttx/.pic = {
    % Draw the transaction box
    \node[rounded corners, minimum width=4.5cm, minimum height=2cm, fill=gray!30, thick, draw=gray]
    (astxbody) at (0,0) {};
    % Draw the label of tx
    \node[above] at (astxbody.north) {\texttt{Assert}};
    % Draw the input
    \node[rounded corners, minimum width=2cm, minimum height=1cm, fill=green!30, thick, draw=green]
    (asinput) at (-1.3,0.2) {$\sigma_{\mathcal{V} + \mathcal{P}}; \sigma_1, \ldots, \sigma_n$ ($d$ \bitcoin)};
    % Draw the outputs
    \node[rounded corners, minimum width=2cm, minimum height=1cm, fill=blue!30, thick, draw=blue]
    (asoutput) at (1.8,0.2) {$d$ \bitcoin};
  },
  payouttx/.pic = {
    % Draw the transaction box
    \node[rounded corners, minimum width=3.5cm, minimum height=1.5cm, fill=gray!30, thick, draw=gray]
    (ptxbody) at (0,0) {};
    % Draw the label of tx
    \node[above] at (ptxbody.north) {\texttt{Payout}};
    % Draw the input
    \node[rounded corners, minimum width=2cm, minimum height=1cm, fill=green!30, thick, draw=green]
    (pinput) at (-1.5,0) {$\sigma_{\mathcal{V} + \mathcal{P}}; \sigma_{\mathcal{P}}$ ($d$ \bitcoin)};
    % Draw output
    \node[rounded corners, minimum width=2cm, minimum height=1cm, fill=blue!30, thick, draw=blue]
    (poutput) at (1.5,0) {$d$ \bitcoin};
    \node[right] at (poutput.east) {$\mathcal{P}$};
  },
  disprovetx/.pic = {
    % Draw the transaction box
    \node[rounded corners, minimum width=5cm, minimum height=2cm, fill=gray!30, thick, draw=gray]
    (dtxbody) at (0,0) {};
    % Draw the label of tx
    \node[above] at (dtxbody.north) {\texttt{Disprove[#1]}};
    % Draw the input
    \node[rounded corners, minimum width=2cm, minimum height=1cm, fill=green!30, thick, draw=green]
    (dinput) at (-1.5,0.2)
    {$\sigma_{\mathcal{V} + \mathcal{P}}; \sigma_{#1}, \sigma_{\the\numexpr#1+1\relax},  z_{#1}, z_{\the\numexpr#1+1\relax}$ ($d$ \bitcoin)};
    % Draw output
    \node[rounded corners, minimum width=2cm, minimum height=1cm, fill=blue!30]
    (doutput1) at (2.5,0.2) {$b$ \bitcoin};
    \node[right] at (doutput1.east) {\texttt{Burn}};
    \node[rounded corners, minimum width=2cm, minimum height=1cm, fill=blue!30, thick, draw=blue]
    (doutput2) at (2.5,-0.9) {$a$ \bitcoin};
    \node[right] at (doutput2.east) {$\mathcal{V}$};

    \draw[dashed, thick, rounded corners] (-3.5,-0.3) rectangle (3.5,0.7);
    % label that this is SINGLE|ANYONECANPAY
    \node[below] at (-2.5,-0.35) {\texttt{SINGLE}};
  },
  disproventx/.pic = {
    % Draw the transaction box
    \node[rounded corners, minimum width=5cm, minimum height=2cm, fill=gray!30, thick, draw=gray]
    (dtxbody) at (0,0) {};
    % Draw the label of tx
    \node[above] at (dtxbody.north) {\texttt{Disprove[n-1]}};
    % Draw the input
    \node[rounded corners, minimum width=2cm, minimum height=1cm, fill=green!30, thick, draw=green]
    (dinput) at (-1.5,0.2)
    {$\sigma_{\mathcal{V} + \mathcal{P}}; \sigma_{n-1}, \sigma_{n},  z_{n-1}, z_{n}$ ($d$ \bitcoin)};
    % Draw output
    \node[rounded corners, minimum width=2cm, minimum height=1cm, fill=blue!30]
    (doutput1) at (2.5,0.2) {$b$ \bitcoin};
    \node[right] at (doutput1.east) {\texttt{Burn}};
    \node[rounded corners, minimum width=2cm, minimum height=1cm, fill=blue!30, thick, draw=blue]
    (doutput2) at (2.5,-0.9) {$a$ \bitcoin};
    \node[right] at (doutput2.east) {$\mathcal{V}$};

    \draw[dashed, thick, rounded corners] (-3.9,-0.3) rectangle (3.5,0.7);
    % label that this is SINGLE|ANYONECANPAY
    \node[below] at (-2.5,-0.35) {\texttt{SINGLE}};
  },
}

\begin{figure}[hb]
  \centering
  \begin{tikzpicture}[scale=0.45, transform shape]
    \pic [at={(0,0)}] {claimtx};

    % create a spending branches as a romb
    \node[rounded corners, fill=black, minimum width=1cm, minimum height=1cm, text=white]
    (cloutput2spending) at ($(cloutput2.east) + (2.5, 0)$) {\texttt{P2TR (key-path)}};

    % connect Claim challenge output and spending branch
    \draw[-{Stealth[length=2mm]},thick] (cloutput2) -- (cloutput2spending);

    % add Challenge tx
    \pic [at={($(cloutput2spending.east) + (4.5,-0.45)$)}] {challengetx};
    % connect Claim output and Challenge input
    \draw[-{Stealth[length=2mm]},red,thick] ($(cloutput2spending.east) - (0,0.25)$) -- (chinput1.west);

    % Add branch for optimistic payout and assert tx
    \node[rounded corners, fill=black, minimum width=1cm, minimum height=1cm, text=white]
    (cloutput1spending) at ($(cloutput1.east) + (2.5,0)$) {\texttt{P2TR (script-path)}};
    % connect Claim output and Spending branch
    \draw[-{Stealth[length=2mm]},thick] (cloutput1) -- (cloutput1spending);

    % Add Optimistic Payout Script box
    \node[rounded corners, fill=blue!70, rotate=90, minimum width=1cm, minimum height=1cm, text=white, thick, draw=blue]
    (clopscript) at ($(cloutput1spending.north) + (-0.75,3)$) {\texttt{OptimisticPayoutScript}};
    % connect spending branch and Optimistic Payout Script
    \draw[-,blue,thick] ($(cloutput1spending.north) + (-0.75,0)$) -- (clopscript.west);

    % Add AssertScript box
    \node[rounded corners, fill=red!50, rotate=90, minimum width=1cm, minimum height=1cm, thick, draw=red]
    (classertscript) at ($(cloutput1spending.north) + (+0.75,2)$) {\texttt{AssertScript}};
    % connect spending branch and Assert Script
    \draw[-,red,thick] ($(cloutput1spending.north) + (+0.75,0)$) -- (classertscript.west);

    
    % draw an optimistic payout transaction
    \pic [at={(14,9)}] {payoutoptimistictx};

    % draw vertical line to connect Claim challenge output and Optimistic Payout input
    \draw[-,blue,thick]
    ($(cloutput2spending.east) - (0,-0.25)$) -- ($(cloutput2spending.east) - (-0.6, -0.25)$);
    \draw[-,blue,thick]
    ($(cloutput2spending.east) - (-0.6, -0.25)$)
    -- ({$(cloutput2spending.east) - (-0.6, -0.25)$} |- {opinput2.west});
    \draw[-{Stealth[length=2mm]},blue,thick]
    ({$(cloutput2spending.east) - (-0.6, -0.25)$} |- {opinput2.west})
    -- (opinput2.west);
    
    
    % draw vertical line from payout optimistic script to later Optimistic Payout input
    \draw[-,blue,thick]
    ($(clopscript.east)$) -- (clopscript.east |- opinput1.west);
    \draw[-{Stealth[length=2mm]},blue,thick] (clopscript.east |- opinput1.west)
    -- node[above] {After Timelock $\Delta_{B}$} (opinput1.west);

    % draw Assert tx
    \pic [at={(chtxbody |- (0,5))}] {asserttx};

    % draw vertical line to connect AssertScript and Assert input
    \draw[-,red,thick]
    ($(classertscript.east)$) -- (classertscript.east |- asinput.west);
    \draw[-{Stealth[length=2mm]},red,thick] (classertscript.east |- asinput.west) -- (asinput.west);

    % draw branch for Assert output
    \node[rounded corners, fill=black, minimum width=1cm, minimum height=1cm, text=white]
    (asoutputspending) at ($(asoutput.east) + (2.5,0)$) {\texttt{P2TR (script-path)}};

    % connect Assert output and Spending branch
    \draw[-{Stealth[length=2mm]},thick] (asoutput) -- (asoutputspending);

    % draw payout script under spending branch for Assert output with small offset
    \node[rounded corners, fill=blue!40, minimum width=1cm, minimum height=1cm, text=white, thick, draw=blue]
    (payoutscript) at ($(asoutputspending.south) + (2,-1)$) {\texttt{PayoutScript}};
    % connect spending branch and Payout Script
    \draw[-,blue!40,thick] ($(asoutputspending.south) - (0.9, 0)$)
    -- ({$(asoutputspending.south) - (0.9, 0)$} |- {payoutscript.west});
    \draw[-{Stealth[length=2mm]},blue!40,thick] ({$(asoutputspending.south) - (0.9, 0)$} |- {payoutscript.west})
    -- (payoutscript.west);

    % draw disprove scripts under spending branch for Assert output with small offset
    \node[rounded corners, fill=red!40, minimum width=1cm, minimum height=1cm, thick, draw=red]
    (dscript1) at ($(payoutscript.south) - (0, 0.6)$) {\texttt{DisproveScript[1]}};

    \node[rounded corners, fill=red!40, minimum width=1cm, minimum height=1cm, thick, draw=red]
    (dscript2) at ($(dscript1.south) - (0, 0.6)$) {\texttt{DisproveScript[2]}};

    \node[rounded corners, fill=red!40, minimum width=1cm, minimum height=1cm, thick, draw=red]
    (dscriptskip) at ($(dscript2.south) - (0, 0.6)$) {$\ldots$};

    \node[rounded corners, fill=red!40, minimum width=1cm, minimum height=1cm, thick, draw=red]
    (dscriptn) at ($(dscriptskip.south) - (0, 0.6)$) {\texttt{DisproveScript[n]}};

    % connect branch and disprove scripts
    \draw[-,red!40,thick] ($(asoutputspending.south) - (1.3, 0)$)
    -- ({$(asoutputspending.south) - (1.3, 0)$} |- {dscriptn.west});
    \draw[-{Stealth[length=2mm]},red!40,thick] ({$(asoutputspending.south) - (1.3, 0)$} |- {dscriptn.west})
    -- (dscriptn.west);
    -- (dscriptskip.west);
    \draw[-{Stealth[length=2mm]},red!40,thick] ({$(asoutputspending.south) - (1.3, 0)$} |- {dscript2.west})
    -- (dscript2.west);
    \draw[-{Stealth[length=2mm]},red!40,thick] ({$(asoutputspending.south) - (1.3, 0)$} |- {dscript1.west})
    -- (dscript1.west);

    % draw payout tx on the same level as Optimistic Payout
    \pic [at={($(optxbody) + (10,0)$)}] {payouttx};
    % connect payout script and payout tx
    \draw[-,blue!40,thick] (payoutscript.east) -- ($(payoutscript.east) + (1,0)$);
    \draw[-,blue!40,thick] ($(payoutscript.east) + (1,0)$)
    -- node[midway, above, sloped] {After timelock $\Delta_{A}$} ({$(payoutscript.east) + (1,0)$}
    |- {pinput.west});
    \draw[-{Stealth[length=2mm]},blue!40,thick] ({$(payoutscript.east) + (1,0)$} |- {pinput.west})
    -- (pinput.west);

    % draw disprove txs
    \pic [at={($(ptxbody) + (1,-2.5)$)}] {disprovetx={1}};
    % draw line from disprove script to disprove tx input
    \draw[-,red!40,thick] (dscript1.east) -- ($(dscript1.east) + (0.9,0)$);
    \draw[-,red!40,thick] ($(dscript1.east) + (0.9,0)$)
    -- ({$(dscript1.east) + (0.9,0)$} |- {dinput.west});
    \draw[-{Stealth[length=2mm]},red!40,thick] ({$(dscript1.east) + (0.9,0)$} |- {dinput.west})
    -- (dinput.west);
    
    \pic [at={($(dtxbody) + (0,-3)$)}] {disprovetx={2}};
    \draw[-,red!40,thick] (dscript2.east) -- ($(dscript2.east) + (1.1,0)$);
    \draw[-,red!40,thick] ($(dscript2.east) + (1.1,0)$)
    -- ({$(dscript2.east) + (1.1,0)$} |- {dinput.west});
    \draw[-{Stealth[length=2mm]},red!40,thick] ({$(dscript2.east) + (1.1,0)$} |- {dinput.west})
    -- (dinput.west);

    \node[rounded corners, minimum width=3cm, minimum height=1cm, fill=gray!30, thick, draw=gray]
    (dtxbody) at ($(dtxbody) + (0,-3)$) {\ldots};
    
    \pic [at={($(dtxbody) + (0,-3)$)}] {disproventx};
    \draw[-,red!40,thick] (dscriptn.east) -- ($(dscriptn.east) + (0.8,0)$);
    \draw[-,red!40,thick] ($(dscriptn.east) + (0.8,0)$)
    -- ({$(dscriptn.east) + (0.8,0)$} |- {dinput.west});
    \draw[-{Stealth[length=2mm]},red!40,thick] ({$(dscriptn.east) + (0.8,0)$} |- {dinput.west})
    -- (dinput.west);
  \end{tikzpicture}
  \caption[BitVM2 Transaction Graph]{BitVM2 transaction graph
   p implementation in \nero. \textcolor{blue}{Blue} --- is
    \texttt{OptimisticPayout} flow, \textcolor{red}{red} --- is start of
    the dispute with \texttt{Challenge} and \texttt{Assert}
    transactions, \textcolor{blue!30}{violet} --- is \texttt{Payout}
    flow, and \textcolor{red!30}{pink} --- is prover punishment.
    $\sigma_{\mathcal{P}}$ --- is provers Schnorr signature,
    $\sigma_{\mathcal{P} + \mathcal{V}}$ --- is Musig2 multisignature of both prover and
    verifier, $\sigma_1, \ldots, \sigma_n$ --- are Winternitz commitments (signatures)
    on intermidiate states $z_1, \ldots, z_n$, $d$ --- staked by prover
    amount, $c$ --- crowdfunded amount for prover to publish Assert tx,
    $b$ --- configurable burn amount ($b < d$), $a$ --- reward for
    punishing prover to verifier}\label{fig:txs-graph}
\end{figure}

\subsection{Claim}\label{sec:claim-tx}

As mentioned earlier, the \texttt{Claim} transaction is the first
transaction in the graph. It is used to state the initial assertion
$f(x) = y$ by publishing $x$ in witness of first input (as $f$ is
assumed known by both parties at start of a session). If
$y \in \{\mathsf{True}, \mathsf{False}\}$, $\mathsf{True}$ is assumed as
expected value, otherwise both $x$ and $y$ are published.

Claim tranasctions has two outputs:

\begin{itemize}
\item \texttt{ClaimAssertOutput} --- \texttt{P2TR} address with two
  alternative spending conditions and unspendable inner key:
  \begin{itemize} 
  \item \texttt{OptimisticPayoutScript} --- checks that timelock
    $\Delta_B$ passed and expects aggregated Musig2 signature from both
    parties.
    \begin{empheqboxed}
      \begin{align*}
        &\elem{\Delta_B} \opcode{OP\_CHECKSEQUANCEVERIFY} \\
        &\elem{\mathsf{pk}_{\mathcal{P} + \mathcal{V}}} \opcode{OP\_CHECKSIG}
      \end{align*}
    \end{empheqboxed}

    where $\mathsf{pk}_{\mathcal{P} + \mathcal{V}}$ --- Musig2 aggregated Schnorr
    $X$-only public key, $\Delta_B$ --- timelock in blocks (the default one
    is around two weeks).
  \item \texttt{AssertScript} --- checks Winternitz commitments
    $\sigma_1, \ldots, \sigma_n$ for intermidiate states $z_1, \ldots, z_n$.

    \begin{empheqboxed}
      \begin{align*}
        & \elem{\mathsf{pk}_{\mathcal{P} + \mathcal{V}}} \opcode{OP\_CHECKSIGVERIFY} \\
        & \elem{\mathsf{pk}_n} \opcode{OP\_WINTERNITZVERIFY} \opcode{OP\_DROP} \\
        & \elem{\mathsf{pk}_{n-1}} \opcode{OP\_WINTERNITZVERIFY} \opcode{OP\_DROP} \\
        & \ldots \\
        & \elem{\mathsf{pk}_0} \opcode{OP\_WINTERNITZVERIFY}
      \end{align*}
    \end{empheqboxed}
    where $\mathsf{pk}_{\mathcal{P} + \mathcal{V}}$ --- Musig2 aggregated Schnorr
    $X$-only public key, $\mathsf{pk}_0, \ldots, \mathsf{pk}_n$ ---
    Winternitz public keys for verification of commitments on
    intermidiate states.
  \end{itemize}
\item \texttt{ClaimChallengeOutput} --- \texttt{P2TR} output with
  \textit{key-path} only spent by \texttt{Challenge} or
  \texttt{OptimisticPayout} transactions.
\end{itemize}

\subsection{Optimistic Payout}\label{sec:optimistic-payout-tx}

The \texttt{OptimisticPayout} transaction is published when timelock
$\Delta_B$ passed and nobody started dispute by publishing
\texttt{Challenge} one, so prover returns staked BTCs back. Note that
the \texttt{OptimisticPayout} is valid only when none of the
\texttt{Challenge} or \texttt{Assert} transactions were published (see
Figure~\ref{fig:txs-graph}).

\texttt{OptimisticPayout} spends both outputs of \texttt{Claim}, for
the first output by \texttt{Optimistic PayoutScript} and using provers
Taproot signature for the second one.

\subsection{Challenge}\label{sec:challenge-tx}

By publishing \texttt{Challenge} transaction verifier starts dispute,
so the \texttt{OptimisticPayout} flow is blocked and the only way to
return provers funds back is by publishing the \texttt{Assert}
tranasction.

As total fee to pay for both \texttt{Assert} and \texttt{Disprove}
transactions is large enough to bankrupt prover in most cases, the
verifier and/or any other external source can help prover by
crowdfunding BTCs in \texttt{Challenge}. That's why \texttt{Challenge}
is signed with both \texttt{SINGLE} and \texttt{ANYONECANPAY}, so
other inputs can be added.

Output amount $c$ in na\"\i{}ve approach is calculated as:

\[
  c = \mathsf{feerate} \cdot (w_{\mathsf{Assert}} + \max_{i \in \{1, \ldots, n\}}{w_{\mathsf{Disprove[i]}}})
\]

where $w_{\mathsf{Assert}}$ and $w_{\mathsf{Disprove[i]}}$ are weights
of \texttt{Assert} and \texttt{Disprove} transactions respectively.

\subsection{Assert, Payout and
  Disprove}\label{sec:assert-payout-disprove-txs}

They are discussed in detail in Section~\ref{sec:assert-tx}. The only
added thing, that both scripts in \texttt{Assert} require aggregated
by both parties Musig2 signatures for covenants emulation, so
\texttt{Payout} and \texttt{Disprove} transaction path them to
fullfill this requirement.

\texttt{Disprove} transaction is signed with \texttt{SINGLE} flag, so
verifiers can add their's outputs to it and receive
$a = d - b- \mathsf{fee}$ difference as a reward for punishment
malicious prover. Also, \texttt{Disprove} uses technic called
``Child-Pays-For-Parent'' which lets \texttt{Disprove} transaction to
spend left Bitcoins as fee for parent \texttt{Assert} and
\texttt{Claim} transactions if need so.

\section{Exploring BitVM2 Potential using Toy
  Examples}\label{sec:covenants-emulation}

Finally, in this section, we explore the potential of \textit{BitVM2}
using some toy
examples. We will consider the following functions:
\begin{itemize}
  \item \textbf{\texttt{u32} Multiplication} --- a function that
    multiplies two 32-bit unsigned integers.
  \item \textbf{Square Fibonacci Sequence} --- a simple function that
    calculates the $n$-th element of the square Fibonacci sequence.
\end{itemize}

We will demonstrate that the current implementation of
\textit{BitVM2} and current
approach to writing Mathematics (finite field arithmetic, elliptic curve
operations etc.) cannot handle even the first example. Based on the second
example, we will show that with the appropriate ideology, the
\textit{BitVM2} can still
be used to verify the execution of complex programs, but written in a different
way. We call such functions as \textbf{BitVM-friendly functions}.
\subsection{u32 Multiplication}

The most basic example is the multiplication of two 32-bit unsigned integers. In
our terminology, $f(x,y) = x \times y$, where the output is a 64-bit unsigned
integer. Since using \texttt{u32} elements in the stack to represent limbs of
the big integer typically results in overflowing, we use two $30$-bit limbs to
represent a \texttt{u32}. This way, the result, being \texttt{u64}, is
represented by three \texttt{u30} limbs. We acknowledge that this is might not
be the most efficient representation, but it should suffice for the
demonstration purposes.

\subsubsection{Implementation Notes}

In this section, we give a brief recap of the BitVM implementation of the big
integer multiplication in Bitcoin Script. Essentially, the Bitcoin script
utilizes the double-and-add method, commonly used for elliptic curve arithmetic.
The idea is following: we can first decompose one of the integers to the binary
form (say, $y={(y_0,\dots,y_{N-1})}_2$ where $N$ is the bitsize of
$y$). Next, we
iterate through each bit and on each step, we double the temporary variable and
add it to the result if the corresponding bit in $y$ is $1$. The concrete
algorithm is described in \Cref{alg:double_and_add}.

\begin{algorithm}
  \caption{Double-and-add method for integer
  multiplication}\label{alg:double_and_add}
  \Input{$x,y$ --- two \texttt{u32} integers being multiplied, $N$
  --- bitsize of $y$.}
  \Output{Result of the multiplication $x \times y$}

  Decompose $y$ to the binary form: ${(y_0,y_1,\dots,k_{N-1})}_2$

  $r \gets 0$

  $t \gets x$

  \For{$i \in \{0,\dots,N-1\}$}{
    \If{$y_i = 1$}{
      $r \gets r + t$
    }

    $t \gets 2 \times t$
  }

  \Return{Integer $r$}

\end{algorithm}

Note that implementing long addition and doubling in Bitcoin Script is quite
cheap, so algorithm turns out to be relatively efficient --- you can read more
in our recently published paper~\cite{w-width-mul}, where we analyze various
strategies of big integer multiplication. In our particular case, we assume that
\texttt{u32} is just a special case of the big integer with the total bitsize of
$N=32$.

\subsubsection{Split Cost Analysis}

The total size of the script turns out to be roughly \textbf{4450 bytes}
(4450B). Now suppose we want to split it into chunks of size $600$. The result
is depicted in \Cref{tab:u32_split}. Note that due to the presence of
\texttt{OP\_IF}'s, we cannot split the program into \textit{exactly} equal parts
of size 600B, so the size insignificantly varies.

\begin{table}[H]
  \centering
  \begin{tabular}{cccc}
    \toprule
    \textbf{Shard number} & \textbf{Shard Size} & \textbf{\# Elements
    in state} & \textbf{Estimated Commitment Cost} \\
    \midrule
    1 & 623B & 37 & 37kB \\
    2 & 640B & 32 & 32kB \\
    3 & 640B & 27 & 27kB \\
    4 & 640B & 22 & 22kB \\
    5 & 640B & 17 & 17kB \\
    6 & 640B & 12 & 12kB \\
    7 & 627B & 3  & 3kB \\
    \bottomrule
  \end{tabular}
  \caption{The result of splitting the program into chunks of
    approximated size $600B$. The second column represents the shard
    size $|f_j|$, the third column number of elements in $z_j$ and
    finally, the last column is the estimated cost of singing
  $z_j$.}\label{tab:u32_split}
\end{table}

Notice an interesting fact: the cost of a single commitment exceeds the cost of
the shard itself many times! For example, if we were to build the
\texttt{DisproveScript} for transition from $z_1$ to $z_2$, we would need the
script of size $37\text{kB}+32\text{kB}+623\text{B} \approx 69.6\text{kB}$! This
leads us to the essential conclusion.

\textbf{Takeaway.} \textit{Optimizing the intermediate states
  representation is crucial for the BitVM2. Even if the program is
split into small chunks, the cost of the commitment can still be overwhelming.}

This leads us to the question: can we throw \textit{BitVM2} out of
the window due to such
inefficiency and wait for the \texttt{OP\_CAT}? The answer is obviously no (for
what other reason are we writing this paper?). We can still use
\textit{BitVM2}, but we
need to change the way we write the programs. We call such programs
\textbf{BitVM-friendly functions}. We provide the first example below.

\subsection{Square Fibonacci Sequence}
Let us consider a toy example of the Square Fibonacci Sequence. Suppose our
input is a pair of elements $(x_0,x_1)$ from the field $\mathbb{F}_q$. For the
sake of convenience, we choose $\mathbb{F}_q$ to be the prime field of BN254
curve, which is frequently used for zk-SNARKs. Then, our program $f_n$ consists
in finding the ${(n-1)}^{\text{th}}$ element in the sequence:
\begin{equation*}
  x_{j+2} = x_{j+1}^2 + x_j^2, \; \text{over $\mathbb{F}_q$.}
\end{equation*}

Such function has a very natural decomposition. Suppose our state is described
by the tuple $(x_j,x_{j+1})$. Consider the transition function $\phi: (x_j,
x_{j+1}) \mapsto (x_{j+1}, x_j^2 + x_{j+1}^2)$. In this case, our function $f_n$
can be defined as:
\begin{equation*}
  f_n = \phi^{(n)}(x_0,x_1)[1],
\end{equation*}

where the index $(a, b)[1] = b$ means the second element in the tuple.

Suppose that we have $\mathtt{Fq}$ implemented in the Bitcoin script. Then, the
state transition function can be implemented as:
\begin{empheqboxed}
  \begin{align*}
    \opcode{\texttt{FQ::DUP}} \, \opcode{\texttt{Fq::SQUARE}}
    \elem{2} \opcode{\texttt{Fq::OP\_ROLL}} \,
    \opcode{\texttt{Fq::SQUARE}} \, \opcode{\texttt{Fq::ADD}}
  \end{align*}
\end{empheqboxed}

The size of this transition is roughly \textit{270 kB} and it requires the
storage of 18 elements in the stack, costing additional \textit{18 kB}. So the
rough size of \texttt{DisproveScript} is \textbf{290 kB}, which is a lot, but
still manageable. In turn, consider the function $f_n$, written in Bitcoin
script:
\begin{empheqboxed}
  \begin{align*}
    &\textbf{for} \; i \in \{1,\dots,n\} \; \textbf{do} \\
    & \;\;\;\; \opcode{\texttt{FQ::DUP}} \,
    \opcode{\texttt{Fq::SQUARE}} \elem{2} \opcode{\texttt{Fq::ROLL}}
    \, \opcode{\texttt{Fq::SQUARE}} \, \opcode{\texttt{Fq::ADD}} \\
    & \textbf{end} \\
    & \opcode{\texttt{Fq::SWAP}} \; \opcode{\texttt{Fq::DROP}}
  \end{align*}
\end{empheqboxed}

For $n=128$, the size is roughly \textbf{35 MB}, which, in contrast, is not at
all manageable. However, the decomposition of the function would make roughly
$n$ scripts, each of size \textbf{290 kB}.

Additionally, notice that regardless of $n$, the size of the disprove scripts is
always the same. Even if we take, say, $n=10000$, making the direct computation
cost roughly $2\text{GB}$, we would have 10000 disprove transactions, each of
size \textbf{290 kB}. Moreover, since the cost of storing the disprove scripts
in the Taptree is negligible, \emph{it does not matter how many chunks we split
the program into}.

\section{Takeaways and Future Directions}

All in all, we believe that, currently, in order to make
\textit{BitVM2} practical, the
whole Groth16 verifier should be written in the \textbf{BitVM-friendly} format.
We give an informal definition below.

\begin{definition}
  A function $f$ is called \textbf{BitVM-friendly} if:
  \begin{itemize}
    \item It can be split into the shards $f_1,\dots,f_n$ of
      relatively small size.
    \item The intermediate states ${\{z_j\}}_{0 \leq j \leq n}$ contain
      a small number of elements, making the commitment cheap enough.
  \end{itemize}

  This way, the worst-case disprove script would cost $\max_{1 \leq j \leq
  n}\left(|f_j| + \alpha(|z_j| + |z_{j-1}|)\right)$ for $\alpha \approx
  1000$\footnote{This constant, after further optimizations, is subject to
  change.}. Note that the number of shards almost does not influence the cost
  since building the larger tree is typically not a problem.
\end{definition}

It is a question, though, whether such BitVM-friendly function exists for all
the Groth16 ingredients. However, we believe that many functions can be
rewritten in such a way. Take, for example, the big integer multiplication. A
great cost of such method is storing the bit decomposition of the number. So, if
we have an $N$-bit integer, the cost of storing the decomposition is roughly
$\alpha N$ (currently, this corresponds to $N \, \text{kB}$). Well, that is a
lot, especially for 254-bit long integers, which are currently used in the
Groth16 verifier. Moreover, there is no efficient way to split the program to
avoid storing the decomposition: you initialize the table at the very beginning
and drop at the very end.

So how to fix this? The answer is simple: manually construct the script so that
at the end of each shard, the decomposition is dropped and $y$ is, therefore,
recovered. Then, at the beginning of the next shard, compute the decomposition
again and proceed. Of course, this would result in the significantly larger $f$,
but the thing is that \textit{we do not care about the total size of $f$, as
long as the commitment size with the shard size is small enough}. Informally, we
present the \Cref{alg:double_and_add_bitvm_friendly} that implements the
double-and-add method in a BitVM-friendly way.

\begin{algorithm}
  \caption{BitVM-friendly double-and-add method}\label{alg:double_and_add}
  \Input{$x,y$ --- two \texttt{u32} integers being multiplied, $N$
  --- bitsize of $y$, $s$ --- parameter to regulate the number of shards.}
  \Output{Result of the multiplication $x \times y$}

  \For{$i \in \{0,\dots,N/s\}$}{
    \textcolor{blue!50!black}{\textbf{Start the shard $i$}}

    Decompose $y$ into the binary form: $y={(y_0,\dots,y_{N-1})}_2$

    \For{$j \in \{0,\dots,s\}$}{
      \If{$y_{is+j} = 1$}{
        $r \gets r + t$
      }

      $t \gets 2 \times t$
    }

    Recover $y$ back to the original form: $y \gets \sum_{i=0}^{N-1} y_i2^i$.

    \textcolor{blue!50!black}{\textbf{End shard $i$}}
  }

  \Return{Integer $r$}
\end{algorithm}\label{alg:double_and_add_bitvm_friendly}

That being said, our future directions are the following:
\begin{itemize}
  \item Try writing the aforementioned algorithm in the BitVM-friendly way.
  \item Experiment whether $w$-width decomposition might make multiplication
    more friendly.
  \item Implement the cost-effective version of the architecture
    (Section 5.3 in~\cite{bitvm2}).
  \item Run the architecture with the simple demo function verification on the
    Bitcoin mainnet.
\end{itemize}

\printbibliography{}

\end{document}
